%%%%%%%%%%%%%%%%%%%%%%%%%%%%%%%%%%%%%%%%%%%%%%%%%%%%%%%%%%%%%%%%%%%%
%% I, the copyright holder of this work, release this work into the
%% public domain. This applies worldwide. In some countries this may
%% not be legally possible; if so: I grant anyone the right to use
%% this work for any purpose, without any conditions, unless such
%% conditions are required by law.
%%%%%%%%%%%%%%%%%%%%%%%%%%%%%%%%%%%%%%%%%%%%%%%%%%%%%%%%%%%%%%%%%%%%



\documentclass{beamer}
\usetheme[faculty=ped]{fibeamer}
\usepackage[utf8]{inputenc}
\usepackage[main=english, czech, slovak]{babel}
\makeatletter % remove the FACULTY OF ... MASARYK UNI LOGO
\renewcommand\fibeamer@includeLogo[1][]{}
\makeatother
\title{Phylogenetics for Predicting Virus Evolution}
\subtitle{Can we anticipate next seasons dominant influenza strain from sequence alignment?}
\author{Paul Wiesemeyer}
\usepackage{ragged2e}  % `\justifying` text
\usepackage{booktabs}  % Tables
\usepackage{tabularx}
\usepackage{tikz}      % Diagrams
\usetikzlibrary{calc, shapes, shapes.geometric, arrows, backgrounds}
\tikzstyle{startstop} = [rectangle, rounded corners, minimum width=3cm, minimum height=1cm, text centered, draw=black, fill=orange]
\usepackage{amsmath, amssymb}
\usepackage{url}       % `\url`s
\usepackage{listings}  % Code listings
\frenchspacing
\begin{document}
  \shorthandoff{-}
  \frame{\maketitle}

    \AtBeginSection[]{% Print an outline at the beginning of sections
  
        \begin{frame}<beamer>
          \frametitle{Outline for Section \thesection}
          \tableofcontents[currentsection]
        \end{frame}
    }
    
  \begin{darkframes}
  \begin{frame}<beamer>
    \frametitle{Problem}
    \begin{itemize}
      \item Influenza evading immunity
      \item Pandemics as with SARS-CoV-2
      \item "Houston, we have aah---CHOO!"
      \item $3 - 11 \% $ of the population catch symptomatic influenza each season. CITATION
      \item Influenza mutates quickly at $\sim 2$ mutations / kilobases / year
      \item No one can tell next seasons circulating influenza strains.
      \item Antigenic shift can cause sudden epidemics and even pandemics.
      \item How can we take informed counter measures on a global level?
    \end{itemize}
  \end{frame}

  \section{Influenza and Vaccines}

    \begin{frame}{\secname}
      \framesubtitle{Influenza---an artful disguise master}
    \end{frame}

    \subsection{Basics}
    \begin{frame}{\subsecname}
      \framesubtitle{Where does influenza come from?}
      \begin{figure}
        \includegraphics[width=\textwidth]{resources/alberts2015molecularp1292.pdf}
        \caption{taken from CITATION Alberts}
      \end{figure}
    \end{frame}

    \subsection{The Hemagglutination Inhibition Assay}
    \begin{frame}{\subsecname}
      \framesubtitle{}
      \begin{figure}
        \includegraphics[width=.7\textwidth]{resources/hemagglutination.pdf}
        \caption{Red blood cells (RBC) precipitate.}
      \end{figure}
    \end{frame}

    \begin{frame}{\subsecname}
      \framesubtitle{}
      \begin{figure}
        \includegraphics[width=.7\textwidth]{resources/hemagglutinationAssay.pdf}
        \caption{Influenza Hemagglutinin (HA) coagulates the RBC, forming a mat.}
      \end{figure}
    \end{frame}

    \begin{frame}{\subsecname}
      \framesubtitle{}
      \begin{figure}
        \includegraphics[width=.7\textwidth]{resources/hemagglutinationInhibitionAssay.pdf}
        \caption{Antisera of the same \textit{serotype} clump the HA, letting the RBC sink to the bottom. This is an (antiserum) concentration dependent process.}
      \end{figure}
    \end{frame}

    \begin{frame}{\subsecname}
      \framesubtitle{}
      \begin{figure}
        \includegraphics[width=.9\textwidth]{resources/HIassaylab.jpg}
        \caption{\footnotesize Here, one antiserum is tested in 12 different dilutions against 8 different virus strains. The highest dilution that prevents agglutination is called the titer.}
      \end{figure}
    \end{frame}



  \section{Phylogenetics}

    \begin{frame}{\secname}
      \framesubtitle{An old idea}
    \end{frame}

    \subsection{The Molecular Clock}
    \begin{frame}{\secname : \subsecname}
      \framesubtitle{}
    \end{frame}

    \subsection{Sequence Alignment}
    \begin{frame}{\secname : \subsecname}
      \framesubtitle{}
    \end{frame}

    \subsection{Building A Phylogenetic Tree}
    \begin{frame}{\secname : \subsecname}
      \framesubtitle{---from sequenced data}
    \end{frame}

  \section{\textit{Nextstrain}}

    \begin{frame}{\secname}
      \framesubtitle{Mending pieces together}
      Please visit nextstrain.org/narratives/
    \end{frame}

    \subsection{How to use the Framework}
    \begin{frame}{\secname : \subsecname}
      \framesubtitle{(this is mirrored in the narrative)}
    \end{frame}

    \subsection{The Powerful Metadata}
    \begin{frame}{\secname : \subsecname}
      \framesubtitle{(this is mirrored in the narrative)}
    \end{frame}

    \subsection{Confidence Levels and Limitations}
    \begin{frame}{\secname : \subsecname}
      \framesubtitle{(this is mirrored in the narrative)}
    \end{frame}

    \begin{frame}{Conclusion}
      \framesubtitle{...and why \textit{Nextstrain} is awesome}
    \end{frame}

    \begin{frame}{References}
      \framesubtitle{}
      \bibliography{phylodynamics.bib}
    \end{frame}

  \end{darkframes}

\end{document}
