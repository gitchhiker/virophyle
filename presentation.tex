%%%%%%%%%%%%%%%%%%%%%%%%%%%%%%%%%%%%%%%%%%%%%%%%%%%%%%%%%%%%%%%%%%%%
%% I, the copyright holder of this work, release this work into the
%% public domain. This applies worldwide. In some countries this may
%% not be legally possible; if so: I grant anyone the right to use
%% this work for any purpose, without any conditions, unless such
%% conditions are required by law.
%%%%%%%%%%%%%%%%%%%%%%%%%%%%%%%%%%%%%%%%%%%%%%%%%%%%%%%%%%%%%%%%%%%%



\documentclass{beamer}
\setbeamertemplate{caption}[numbered]


\usetheme[faculty=ped]{fibeamer}
\usepackage[utf8]{inputenc}
\usepackage[main=english, czech, slovak]{babel}
\makeatletter % remove the FACULTY OF ... MASARYK UNI LOGO
\renewcommand\fibeamer@includeLogo[1][]{}
\makeatother
\title{Phylogenetics for Predicting Virus Evolution}
\subtitle{Anticipating next seasons influenza strains}
\author{Paul Wiesemeyer}
\usepackage{ragged2e}  % `\justifying` text
\usepackage{booktabs}  % Tables
\usepackage{tabularx}
\usepackage{tikz}      % Diagrams
\usetikzlibrary{calc, shapes, shapes.geometric, arrows, backgrounds}
\tikzstyle{startstop} = [rectangle, rounded corners, minimum width=3cm, minimum height=1cm, text centered, draw=black, fill=orange]
\usepackage{amsmath, amssymb}

\usepackage{url}       % `\url`s
\usepackage{listings}  % Code listings
\frenchspacing

%\usepackage[backend=biber,style=alphabetic,citestyle=authoryear]{biblatex}
%\addbibresource{virophyle.bib}

%      \bibliographystyle{unsrt}

%      \bibliography{virophyle}

%\usepackage[plainnat]{natbib}


\begin{document}
  \shorthandoff{-}
  \frame{\maketitle}

    % \AtBeginSection[]{% Print an outline at the beginning of sections
  
    %     \begin{frame}<beamer>
    %       \frametitle{Outline for Section \thesection}
    %       \tableofcontents[currentsection]
    %     \end{frame}
    % }
    
  \begin{darkframes}




  \begin{frame}<beamer>
    \frametitle{Problem}

    \begin{tikzpicture}[remember picture,overlay]
      \node[xshift=-4cm,yshift=-2cm] at (current page.north east) {\includegraphics[width=.7\textwidth]{resources/Houston.pdf}};
    \end{tikzpicture}

    \begin{itemize}
      \itemsep1em
      \item Estimated $3 - 11 \% $ of the population catch symptomatic influenza each season \cite{tokarsOlsen+18a}
      \item Influenza evolves rapidly and evades immunity and vaccines
      \item Epidemics and pandemics caused by RNA recombination events
      \item Science is slow responding to health issues
    \end{itemize}
  \end{frame}

  \begin{frame}<beamer>
    \frametitle{Problem}
    \begin{itemize}
      \item $\sim$ \textit{8 months} from submission to publication of a medical paper \cite{aAMC18}
      \item $\sim$ \textit{6 months} from vaccine strain selection to distribution:
    \end{itemize}
    \begin{figure}
        \includegraphics[width=\textwidth]{resources/flu_vaccine_schedule.png}
        \caption{\cite{bedford15}}
    \end{figure}{}
  \end{frame}





  \begin{frame}<beamer>
           \frametitle{Outline}
           \footnotesize
           \tableofcontents
  \end{frame}







  \section{Influenza}

    \begin{frame}{\secname}
      \framesubtitle{---a closer look}
      \begin{tikzpicture}
        \usebeamercolor{fibeamer}
        \node [anchor=west] (rna) at (-1,6.5) {\footnotesize 8 single stranded RNA};
        \node [anchor=west] (ha) at (-1,1) {\footnotesize Hemagglutinin (HA)};
        \node [anchor=west] (na) at (-1,3) {\footnotesize Neuraminidase (NA)};
        \begin{scope}[xshift=1.5cm]
          \node[anchor=south west,inner sep=0] (image) at (0,0) {\includegraphics[width=0.8\textwidth]{resources/flu-JPG.png}};
          \begin{scope}[x={(image.south east)},y={(image.north west)}]
            \draw [-latex, thick, orange] (rna) to[out=0, in=150] (0.48,0.90);
            \draw [-latex, thick, orange] (ha) -- ++(0.70,0.0);
            \draw [-latex, thick, orange] (na) -- ++(0.4,0.0);
          \end{scope}
        \end{scope}
      \end{tikzpicture}
      % \begin{figure}
      %   \includegraphics[width=.8\textwidth]{resources/flu-JPG.png}
      %   \caption{\footnotesize \cite{cDC20}}
      % \end{figure}
    \end{frame}




    \begin{frame}{\secname}
      \framesubtitle{---the recent history}
      \begin{figure}
        \includegraphics[width=\textwidth]{resources/alberts2015molecularp1292.pdf}
        \caption{taken from \cite{alberts15}}
      \end{figure}{}
    \end{frame}{}

    \begin{frame}{\secname}
      \framesubtitle{---the phylogeny}
      \begin{figure}
        \includegraphics[width=.7\textwidth]{resources/h3n2escape.pdf}
        \caption{\footnotesize }
      \end{figure}
    \end{frame}

    \begin{frame}{\secname}
      \framesubtitle{A virus that does not change escape vehicle will get caught}
      \begin{figure}
        \includegraphics[width=.9\textwidth]{resources/GTAsa.jpg}
        \caption{\footnotesize It's like getting more and more stars, having to switch escape vehicle constantly}
      \end{figure}
    \end{frame}




  \section{Phylogenetics}

    \begin{frame}{\secname}
      \framesubtitle{An old idea}
      Haeckel, Darwin
    \end{frame}

    \subsection{The Molecular Clock}

    \begin{frame}{\secname}
      \framesubtitle{Concept: \subsecname}
      \begin{figure}
        \includegraphics[width=\textwidth]{resources/molClock1.pdf}
        \caption{\footnotesize Linear time-mutation relationship}
      \end{figure}
    \end{frame}


    \begin{frame}{\secname}
      \framesubtitle{Concept: \subsecname}
      \begin{figure}
        \includegraphics[width=\textwidth]{resources/molClock2.pdf}
        \caption{\footnotesize Linear time-mutation relationship}
      \end{figure}
      \vspace*{-.2cm}
      Corroborated by genetic equidistance.

      Limited by complete turnover time.
    \end{frame}

    \subsection{Sequence Alignment}

    \begin{frame}{\subsecname}
      \framesubtitle{An example algorithm}

    \end{frame}

    \subsection{Inferring the tree}

    \begin{frame}{\subsecname}
      \framesubtitle{Basic Principles}

      Parsimony --- Ockhams razor, Cost function

    \end{frame}

    \subsection{Predicting Virus Evolution}

    \begin{frame}
      \framesubtitle{Approaches}

      Epidemiological: Is there a geographical region?

      Genealogical: Is there a certain type of Mutation?

      Immunological: Hemagglutinin inhibition cartography

      (Denote the scales here: Molecule/Cell/Population)

      Vision: Bring these levels together
    \end{frame}

    \subsection{The Hemagglutination Inhibition Assay}

    \begin{frame}{\subsecname}
      \framesubtitle{}
      \begin{figure}
        \includegraphics[width=.7\textwidth]{resources/hemagglutination.pdf}
        \caption{Red blood cells (RBC) precipitate.}
        \label{1}
      \end{figure}
    \end{frame}

    \begin{frame}{\subsecname}
      \framesubtitle{}
      \begin{figure}
        \includegraphics[width=.7\textwidth]{resources/hemagglutinationAssay.pdf}
        \caption{Influenza Hemagglutinin (HA) coagulates the RBC, forming a mat.}
      \end{figure}
    \end{frame}

    \begin{frame}{\subsecname}
      \framesubtitle{}
      \begin{figure}
        \includegraphics[width=.7\textwidth]{resources/hemagglutinationInhibitionAssay.pdf}
        \caption{Antisera of the same \textit{serotype} clump the HA, letting the RBC sink to the bottom. This is an (antiserum) concentration dependent process.}
      \end{figure}
    \end{frame}

    \begin{frame}{\subsecname}
      \framesubtitle{}
      \begin{figure}
        \includegraphics[width=.9\textwidth]{resources/HIassaylab.jpg}
        \caption{\footnotesize Here, one antiserum is tested in 12 different dilutions against 8 different virus strains. The highest dilution that prevents agglutination is called the titer.}
      \end{figure}
    \end{frame}

    \begin{frame}{\subsecname}
      \framesubtitle{How this used to be looked at}
      % \begin{figure}
      %   \centering
      %   \begin{subfigure}{.5\textwidth}
      %     \centering
      %     \includegraphics[width=\linewidth]{resources/HItable.png}
      %     \caption{A subfigure}
      %     \label{fig:sub1}
      %   \end{subfigure}%
      %   \begin{subfigure}{.5\textwidth}
      %     \centering
      %     \includegraphics[width=.4\linewidth]{resources/hi_smith_map.jpg}
      %     \caption{A subfigure}
      %     \label{fig:sub2}
      %   \end{subfigure}
      %   \caption{A figure with two subfigures}
      %   \label{fig:test}
      %   \end{figure}
    \end{frame}

    \begin{frame}{\subsecname}
      \framesubtitle{}
      \begin{figure}
        \includegraphics[width=\textwidth]{resources/HItableToTree.pdf}
        \caption{\footnotesize Mapping the chart to the tree constructed from sequences}
      \end{figure}
    \end{frame}

    \begin{frame}{\subsecname}
      \framesubtitle{}
      \begin{figure}
        \includegraphics[width=\textwidth]{resources/HItableToTreeDiffs.pdf}
        \caption{\footnotesize Here, one antiserum is tested in 12 different dilutions against 8 different virus strains. The highest dilution that prevents agglutination is called the titer.}
      \end{figure}
    \end{frame}

    \begin{frame}{\subsecname}
      \framesubtitle{}
      \begin{figure}
        \includegraphics[width=\textwidth]{resources/HItableToTreeUru.pdf}
        \caption{\footnotesize Here, one antiserum is tested in 12 different dilutions against 8 different virus strains. The highest dilution that prevents agglutination is called the titer.}
      \end{figure}
    \end{frame}

    \begin{frame}{\subsecname}
      \framesubtitle{\small The asymmetry makes sense, think of it like this but with more dmensions}
      \begin{figure}
        \includegraphics[width=.7\textwidth]{resources/shapesOfWood.jpg}
        \caption{\footnotesize \cite{rosipaw10}}
      \end{figure}
    \end{frame}

    \begin{frame}[allowframebreaks]{Formulas}
      \framesubtitle{}
      $T_{a\beta}~\ $........................... HI titer of virus $a$ against antiserum $\beta$ (virus $b$)

      $H_{a\beta}~$........................... $\log_2$ relative titer (we'll use this one)
      \begin{align}
        H_{a\beta} = \log_2 ~ T_{b\beta} - \log_2 ~ T_{a\beta}
      \end{align}

      $\hat{H}_{a\beta}~$........................... predicted $\log_2$ relative titer
      $v_{a}~~~$........................... avidity of virus $a$ (=greediness)
      $p_{\beta}~~~$........................... potency of antiserum $\beta$ (=effectiveness)
      $D_{a\beta}~$........................... genetic component of titer drop

      \begin{align}
        \hat{H}_{a\beta} = v_a + p_\beta + D_{ab}
      \end{align}


      What remains is split up into a sum over individual mutation contributions:
      \begin{align}
        D_{ab} = \sum_{i \in (a ... b)} d_i
      \end{align}
      Where the sum is over the path connecting virus $a$ and virus $ b \stackrel{}={}$ antiserum $\beta$. Now we want
      \begin{align}
        \hat{H}_{a\beta} \stackrel{!}={} {H}_{a\beta}
      \end{align}
      so we minimize a cost function $C$ of the whole tree:
      \begin{align}
        C := \sum_{a,\beta}( \hat{H}_{a\beta} - {H}_{a\beta})^2 + \lambda \sum_i d_i + \gamma \sum_a v_a^2 + \delta \sum_\alpha p_\alpha^2
      \end{align}

    \end{frame}













  \section{Predicting the next strain of influenza}

  \subsection{Results}



    \begin{frame}{\subsecname}
      \framesubtitle{tree model}
      \begin{figure}
        \includegraphics[width=\textwidth]{resources/neher+16/tree_model.pdf}
        \caption{\footnotesize a}
      \end{figure}
    \end{frame}

    \begin{frame}{\subsecname}
      \framesubtitle{tree model hist}
      \begin{figure}
        \includegraphics[width=\textwidth]{resources/neher+16/tree_model_hist.pdf}
        \caption{\footnotesize a}
      \end{figure}
    \end{frame}

    \begin{frame}{\subsecname}
      \framesubtitle{recent antigenic evolution}
      \begin{figure}
        \includegraphics[width=\textwidth]{resources/neher+16/recent_antigenic_evolution.pdf}
        \caption{\footnotesize a}
      \end{figure}
    \end{frame}

    \begin{frame}{\subsecname}
      \framesubtitle{fraction frequency threshold color bar}
      \begin{figure}
        \includegraphics[width=.8\textwidth]{resources/neher+16/fraction_frequency_threshold_colourbar.pdf}
        \caption{\footnotesize a}
      \end{figure}
    \end{frame}

    \begin{frame}{\subsecname}
      \framesubtitle{cumulative antigenic change}
      \begin{figure}
        \includegraphics[width=\textwidth]{resources/neher+16/cumulative_antigenic_change.pdf}
        \caption{\footnotesize a}
      \end{figure}
    \end{frame}
    
    \begin{frame}{\subsecname}
      \framesubtitle{distance to season year}
      \begin{figure}
        \includegraphics[width=.8\textwidth]{resources/neher+16/distance_to_season_year.pdf}
        \caption{\footnotesize a}
      \end{figure}
    \end{frame}


    \begin{frame}{\subsecname}
      \framesubtitle{years antigenic advance}
      \begin{figure}
        \includegraphics[width=.7\textwidth]{resources/neher+16/years_antigenic_advance.pdf}
        \caption{\footnotesize a}
      \end{figure}
    \end{frame}









  \section{\textit{Nextstrain}}

    \begin{frame}{\secname}
      \framesubtitle{Mending pieces together}
      Please visit nextstrain.org/narratives/
    \end{frame}

    \subsection{How to use the Framework}
    \begin{frame}{\secname : \subsecname}
      \framesubtitle{(this is mirrored in the narrative)}
      \begin{figure}
        \includegraphics[width=\textwidth]{resources/NextstrainPathogens.pdf}
        \caption{\footnotesize }
      \end{figure}
    \end{frame}

    \subsection{The Powerful Meta Data}
    \begin{frame}{\secname : \subsecname}
      \framesubtitle{(this is mirrored in the narrative)}
    \end{frame}

    \subsection{Confidence Levels and Limitations}
    \begin{frame}{\secname : \subsecname}
      \framesubtitle{(this is mirrored in the narrative)}
    \end{frame}

    \begin{frame}{Multi Scale Evolution}
      \framesubtitle{}
      If a single event mutation occurs, say \textbf{D 186 G} in the HA genome, it is subject to multi scale evolutionary selection:
      \begin{itemize}
        \item this RNA instance vs. the other RNA strands in the same cell
        \item this cell's mutated viruses vs. other viruses inside the host
        \item this hosts viruses vs. viruses in rest of the population
        \item this population vs. other populations
      \end{itemize}
      These scales are difficult to separate. At the population level \textit{epidemiological} processes may dominate.
    \end{frame}

    \begin{frame}{Conclusion}
      \framesubtitle{...and why \textit{Nextstrain} is awesome}
    \end{frame}

    \begin{frame}[allowframebreaks]{References}
    \tiny
\bibliography{virophyle}
\bibliographystyle{apalike}
%\printbibliography
    \end{frame}




  \end{darkframes}

\end{document}
