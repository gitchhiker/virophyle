\documentclass[12pt]{scrartcl}

\author{Paul Wiesemeyer}

\title{Phylogenetics\texorpdfstring{\\}{,} for Predicting Virus Evolution}

\subtitle{Including a brief guide to the open science tool {\LARGE\textit{nextstrain.org}} \texorpdfstring{\\[1cm]}{,}A Contribution to the 2020 Seminar ``Physics of Viruses'' Conducted by Ulrich Schwarz and Frederik Graw at University Heidelberg }

\date{\today}


\input{latex-includes/includes}
\usepackage[square]{natbib}

\usepackage{tikz}
\usetikzlibrary{trees}
\usetikzlibrary{shapes}
\usepackage{amsmath}
\usepackage{xspace}
\usepackage{forest}
\usepackage{hyperref}

\usepackage{graphicx}
\usepackage{lmodern}

\usepackage[acronym]{glossaries}
\renewcommand{\glstextformat}[1]{\textit{#1}}
\makeglossaries
\newglossaryentry{hemagglutinationAssay}
{
    name=Hemagglutination Assay,
    description={A Lab experiment that classifies Influenza type A viruses according to their hemagglutinin surface protein}
}

\newglossaryentry{openScience}
{
    name=open science,
    description={The development to make scientific results, data and alike easily accessible to the public}
}

\newglossaryentry{phylogenetics}
{
    name=Phylogenetics,
    description={The science that extracts information from genetic sequences by classifying their inter-relatedness and inferring a \acrshort{ml} tree}
}

\newglossaryentry{phylodynamics}
{
    name=Phylodynamics,
    description={The branch of science that analyzes the various processes shaping phylogenies. These usually comprise evolutionary feedbacks as well as shorter timescale processes such as an extinction event}
}

\newglossaryentry{phylogeny}
{
    name=phylogeny,
    description={another word for phylogenetic tree}
}

\newglossaryentry{influenza}
{
    name=Influenza,
    description={The disease that is caused by one of the influenza viruses. Also called flu}
}

\newglossaryentry{evolution}
{
    name=Evolution,
    description={Systems that have a sense of succession and contain some form of \gls{heredity}, \gls{variation} and \gls{selection} show evolution. This can be as simple as ``survival of the fittest'' but also---depending on system complexity---a wide plethora of other dynamics}
}

\newglossaryentry{variation}
{
    name=variation,
    description={In \gls{evolution}, variation refers to the stochasticity of the properties that one generation inherits from its precursor}
}

\newglossaryentry{heredity}
{
    name=heredity,
    description={In \gls{evolution}, heredity means that one generation inherits traits from its preceding generation}
}

\newglossaryentry{selection}
{
    name=selection,
    description={In \gls{evolution}, selection is the feedback between fitness and probability to produce a next generation}
}

\newglossaryentry{strandedness}
{
    name=strandedness,
    description={In genetics, strandedness refers to whether a nucleic acid consists of one or two strands. The latter usually provides more stability, i.e. higher fidelity when copied}
}

\newglossaryentry{sense}
{
    name=sense,
    description={In genetics, the sense of a nucleic acid, indicates the roles of the strand (and its complement) in specifying a sequence of amino acids. In virology, positive-sense viral RNA means that the viral RNA sequence may be directly translated into viral proteins (5'-to-3')}
}

\newglossaryentry{capsid}
{
    name=capsid,
    description={The capsid of a virus is the protein shell that encloses the genetic material. It usually consists of many so called protomers that form an almost spherical structure (platonic body)}
}

\newglossaryentry{bilipidLayer}
{
    name=bilipid layer,
    description={The most ubiquitous membrane found in nature that consists of a double layer of lipid molecules}
}

\newglossaryentry{epitope}
{
    name=epitope,
    description={The part of an \gls{antigen}, such as a viral surface protein, that is recognized by the immune system and that antibodies bind to}
}

\newglossaryentry{antigen}
{
    name=antigen,
    description={Any kind of structure, that provokes a reaction by the immune system. See \gls{epitope}}
}

\newglossaryentry{immunology}
{
    name=immunology,
    description={Branch of science that deals with the immune system of organisms}
}

\newglossaryentry{epidemiology}
{
    name=epidemiology,
    description={Branch of science that studies the spreading and distribution of diseases and other health factors in populations. This can range from the abstract formulation of governing equations all the way to the exact description of a particular epidemic such as COVID-19}
}

\newglossaryentry{strain}
{
    name=strain,
    description={In biology, a strain refers to a subtype or a variant of some species.}
}

\newglossaryentry{species}
{
    name=species,
    description={In biology, ``species'' is the most widely used classification term with a range of definitions. Traditionally defined as ``largest group of organisms that are capable of producing fertile offspring'' it is used even beyond the realms of sexual reproduction with more sophisticated methods---e.g. based on genomic data---to distinguish between species}
}

\newglossaryentry{serotype}
{
    name=serotype,
    description={In biology, ``species'' is the most widely used classification term with a range of definitions. Traditionally defined as ``largest group of organisms that are capable of producing fertile offspring'' it is used even beyond the realms of sexual reproduction with more sophisticated methods---e.g. based on genomic data---to distinguish between species}
}

\newglossaryentry{titer}
{
    name=titer,
    description={SOMETHING SOMETHING}
}

\newglossaryentry{antigenicAdvancement}
{
    name=antigenic advancement,
    description={SOMETHING SOMETHING}
}

\newglossaryentry{antigenicDrift}
{
    name=antigenic drift,
    description={SOMETHING SOMETHING}
}

\newglossaryentry{antigenicShift}
{
    name=antigenic shift,
    description={SOMETHING SOMETHING}
}

\newglossaryentry{pointMutation}
{
    name=point mutation,
    description={SOMETHING SOMETHING}
}

\newglossaryentry{indel}
{
    name=indel,
    description={SOMETHING SOMETHING}
}

\newglossaryentry{allele}
{
    name=allele,
    description={SOMETHING SOMETHING}
}

\newacronym{rbd}{RBD}{Receptor Binding Domain}

\newacronym{hi}{HI}{Hemagglutination Inhibition}

\newacronym{ha}{HA}{Hemagglutinin}

\newacronym{na}{NA}{Neuraminidase}

\newacronym{rna}{RNA}{Ribonucleic Acid}

\newacronym{dna}{DNA}{Deoxyribonucleic Acid}

\newacronym{ml}{ML}{Maximum Likelihood}






\newcommand{\red}[1]{{\color{red}#1}}
\newcommand{\gray}[1]{{\color{gray}#1}}



\begin{document}

\maketitle

\clearpage

\section*{Aim of this write-up}

  After reading these pages\textemdash and potentially following the recommended material at the end of each section\textemdash the reader will be able to:
  \begin{itemize}
    \item Read phylogenetic trees
    \item Explore the framework nextstrain.org
    \item Have some in-detail knowledge about influenza
    \item Have gotten a close look at the phylogenetic approach at prediction of influenza evolution
    \item ... and hopefully end up with a sprawling interest in virus research!
  \end{itemize}

  As the research on this vast field proved cumbersome due to many terms and definitions unfamiliar to a physics graduate, this write-up is augmented by a glossary and a list of acronyms, found at the very end.

\vfill

\tableofcontents

\clearpage

\section{Resumée}



  \subsection*{Old Introduction}
    \gray{

      Viruses, and the diseases they provoke, put a large burden on human society. \red{Viruses are ...(TODO) .} comparably simple structures are capable of modifying and possibly destroying vital functions in the human body, by infiltrating their genetic material into the reproductory apparatus of cells.

      All viruses that persist in the human population over longer times share some common features:
      \begin{itemize}
        \item They need some form of protection from their surroundings, a hull
        \item They need to find a way to get into human cells
        \item They aim at reproducing quickly
        \item If there is no constant infection source, they need to find a path to get from one individual to another
        \item They will have to deal with human immune system response
        \item They will underlie some kind of evolutionary pressure (too general).
      \end{itemize}

      (This rough characterization is of course incomplete, and there may also be exceptions. The terms ``need'' and ``aim at'' are to be understood as: evolutionary processes strongly favor these characteristics.)

      To give another rough image of what makes up a virus, here are the essential constituents:
      \begin{itemize}
        \item A piece of genetic code, most importantly categorized into: \acrfull{rna} or \acrfull{dna}, single or double stranded, positive or negative sense, length (usually some few kilo base (pairs) long)
        \item A hull: either only a capsid (of proteins) or an additional envelope (bilipid layer).
      \end{itemize}

      For most enveloped viruses such as members of the \textit{Orthomyxoviridae} family (including influenza) or \textit{Coronaviridae} (e.g. SARS-CoV-2), there is a variety of different surface proteins that populate the bi-lipid layer.

      These surface proteins take up functions such as binding to a cell to infiltrate the viruses genetic sequence into it, or releasing a freshly assembled virus from the host cell's surface into the surrounding body fluids. This is usually done via interactions with specific host cell receptors, therefore defining a \acrfull{rbd} as the part of the surface protein that fits onto the receptor like a key fits into a lock.

      At the same time, the human immune system will also interact with mostly surface proteins. The immune system will eventually develop antibodies that are targeted to bind to a specific region of virus surface proteins, thereby rendering them innocuous. The surface protein region targeted by the immune system is called \textit{epitope site} and may have a large overlap with the \acrshort{rbd}.

      These two mechanisms put an evolutionary pressure on the virus, especially on the surface proteins and its epitope sites. While having to maintain essential functioning such as binding to receptors, the virus will draw large advantage from modifying its epitope site to such an extent that the antibodies cannot bind to it any longer. This way of disguising itself\textemdash by amino acid mutations in crucial places\textemdash will allow the virus to reinfect previously immune individuals of the host population.

      An accelerated evolution therefore allows viruses to persist in the human population over long periods, but at the same time provides a large record to trace its spreading history, when the viral genome is sequenced and the information is curated.

      In this write-up, we will look at how sequencing of the viral genome, along with phylogenetic tree inference, can provide useful insights on (i) the route that the virus takes to spread in the human population, (ii) the prevalence of strains within a virus type, and (iii) the (projected) evasion of the virus from immune system response and vaccines.

      \red{Vaccines need updates. Scientific publishing and vaccine manufacturing take in the order of half a year each. Two influenza seasons (at a given place)are usually only a few months apart.}

      The goal of the \textit{nextstrain} research group is to make phylogenetic trees inferred from external sequencing labs quickly accessible and easily explorable for fellow scientists, health care officials and the public. This is urgently needed in highly dynamic situations such as the 2020 COVID-19 pandemic. Also, \textit{nextstrain} provides an integrated visualization of pathogen spreading phenomena, as will be explained in section \ref{nextstrain}.

      Here, we chose to look at influenza, as it is the evergreen. \red{Recombination pathway for evo.}

    } % grey

    \subsection{What are viruses?}

    History: infectious fluids w/o bacteria. Turns out the mere information can modify a living being, using its reproductory apparatus. Viruses also mark the border to what is considered the living world, being classified as below it.

    Functional definition: Genetic material. Mechanism to enter ``hijack'' host cell. Often: protective hull.

    Detailed definition (box?): Different types of genetic code. Different types of protection from the outer world. Different mechanisms to enter a cell.

    Include: Spreading of viruses. Airborne, Waterborne etc. most importantly: directly host-to-host, or does it need a source or intermediary host or reservoir?

    Organizational structure of human society makes up a big threat when it comes to human-to-human transmission. What used to be a spatial problem in past centuries (e.g. the wavelike patterns by the bubonic plague) has now turned into a highly interconnected world, that is the structure of the population has changed dramatically. This accelerates pathogen transmission, as it ... but it also open new opportunities for countermeasures.

    \subsection{What is problematic when viruses evolve?}

    Evade immune system. And even vaccines. Leads to a race and so called coevolution. Mention the Red Queen here.

    But one big advantage: We can now (since when?) sequence the genetic material and through the statistical nature of its modifications (mutations) extract information from it.

    Most importantly we can interrelate the different probes. Where we just had 100 people with same symptoms, we can now make precise statements about who probably got the virus from whom. With a certain uncertainity.

    So how can we make sense of this?

    The theory behind this is called phylogenetics.



\section{Phylogenetics}

  

  Essentially the Wikipedia Article + Volz 2013 (TODO read)

  \subsection{Idea}

  \Gls{phylogenetics} is the

  \subsection{How to Read a Phylogenetic Tree}

  \subsection{Recommended Material}




\section{Influenza}

  Influenza is one of the most common viral diseases \citep{DudaMenna20} and puts a large burden on human health. The disease is estimated to cause symptomatic illness in about $3 - 11 \%$ of the human population (in the U.S.) \citep{tokarsOlsen+18} every season. Globally, up to $650,000$ individuals die from influenza illness every season \cite[see][]{iulianoRoguski+18}.

  Influenza refers to a disease caused by a virus of the influenza family \textit{Orthomyxoviridae}. All \textit{orthomyxo} viruses have an envelope that carries surface proteins and their genome is a negative sense RNA.

  Most relevant for human infectious diseases are the genera \textit{Alpha-} and \textit{Betainfluenzavirus} that contain the species \textit{Influenza A} and \textit{Influenza B} respectively.

  They are further subdivided into their so called \textit{Serotype}, a classification by coagulation behavior in the \gls{hemagglutinationAssay}\\[0mm]


  \begin{tabular}{ c c c c c }

   Taxonomy: & \hspace{18mm}Family\hspace{13mm} & Genus\hspace{3mm} & Species\hspace{10mm} & Serotype\hspace{1mm}\\
    
  \end{tabular}\\[0mm]

  \begin{forest}
    for tree={
      grow'=east,
      draw,
      calign=first,
      font=\sffamily,
      rounded corners,
      parent anchor=east,
      child anchor=west,
      edge path={%
        \noexpand\path [\forestoption{edge}] (!u.parent anchor) -- ++(0pt,0) |- (.child anchor)\forestoption{edge label};
      }
    },
    gray/.style={
      draw=none,
      color=gray
    }
    [{Viruses}
      [{$\cdot\cdot\cdot$}, gray
        [{\textit{Orthomyxoviridae}},for tree={l sep= 2mm}
          [{$\alpha$},for tree={l sep= 12mm}
            [{A},for tree={l sep= 18mm}
              [{H1N1}]
              [{H3N2}]
            ]
          ]
          [{$\beta$},for tree={l sep= 12mm}
            [{B},for tree={l sep= 18mm}
              [{Yamagata}]
              [{...}, gray]
            ]
          ]
          [{... (5 more)}, gray]
        ]
        [{\textit{... (149 more)}}, gray, for tree={l sep= 2mm}]
      ]
    ]
  \end{forest}

\red{Recombination evo pathway. E.g. Avian H7N9 Flu CFR 15-60 \% but no H2H}


  \subsection{Basics}

    History diagram from \citep{alberts15}

    HA: epl

    NA: expl

    Phylo tree of A/H3N2 HA Volz+2013

  \subsection{Hemagglutination Inhibition Assay}

    Short description

    Hirst 1943

    mention cartography from Smith+2004 (?) 

\section{Nextstrain} \label{nextstrain}

  In February 2017, a research tool, consisting

  \subsection{Idea}

  \subsection{Short Usage Instructions}
  \begin{figure}[h!]
    \makebox[\textwidth][c]{
    \includegraphics[width=1.2\textwidth]{linked/screen_red_boxes_next_flu_2y.pdf}}
    \caption{\footnotesize Taken from the \href{https://nextstrain.org/flu/seasonal/h3n2/ha/2y?p=grid}{2 year seasonal H3N2 flu HA dataset visualization} on \href{https://nextstrain.org}{\textbf{nextstrain.org}}, \cite{leeMoncla+20}}
  \end{figure}


  \subsection{Example}

  \subsection{Recommended Material}

\section{Mapping Influenza Evolution}

  \subsection{Mapping Titer to Tree}

    Minimizing a Cost function

    Tree model vs. substitution model

    Proving Treelikeness

  \subsection{Results}

  \begin{figure}[h!]
    \includegraphics[width=\textwidth]{linked/neher+16/recent_antigenic_evolution.pdf}
    \caption{\footnotesize For high recent antigenic evolution traits, $25\%$ prevalence directly entails $75\%$ \cite{neherBedford+16}}
  \end{figure}

  \begin{figure}[h!]
    \includegraphics[width=.8\textwidth]{linked/neher+16/fraction_frequency_threshold_colourbar.pdf}
    \caption{\footnotesize For high recent antigenic evolution traits, $25\%$ prevalence directly entails $75\%$ \cite{neherBedford+16}}
  \end{figure}

    Figure 6 Interpretation

\section{Conclusion}

  Veryasfsadffds interestidsfng! \gls{maths} \acrfull{gcd} sdfg \gls{latex}

\clearpage


\section{References, Acronyms, and Glossary}


\bibliography{virophyle}
\bibliographystyle{apalike}

\printglossary[type=\acronymtype]

\printglossary

% 
\section{Introduction}

\subsection{Problem}

Viruses and the diseases they provoke put a large burden on human life. These comparably simple structures are capable of modifying and possibly destroying vital functions in the human body, by infiltrating their genetic material into the reproductory apparatus of cells.

All viruses that persist in the human population over longer times share some common features:
\begin{itemize}
  \item They need some form of protection from their surroundings, a hull
  \item They need to find a way to get into human cells
  \item They aim at reproducing quickly
  \item If there is no constant infection source, they need to find a path to get from one individual to another
  \item They will have to deal with human immune system response
  \item They will underlie some kind of evolutionary pressure
\end{itemize}

(This rough characterisation is of course incomplete, and there may also be exceptions.)

To give another rough image of what makes up a virus, here are the essential constituents:
\begin{itemize}
  \item A piece of genetic code (RNA or DNA, positive or negative sense, usually some few kilobase(pairs) long)
  \item A hull: either only a capsid (of proteines) or an additional envelope (bilipid layer) with surface proteines
\end{itemize}

For enveloped viruses such as in the \textit{Orthomyxoviridae} family (including influenza) or \textit{Coronaviridae} (e.g. SARS-CoV-2), there is usually a variety of different surface proteins that populate the bilipid layer.

These surface proteins take up functions such as binding to a cell to infiltrate the viruses genetic sequence into it, or releasing a freshly assembled virus from the host cell's surface into the surrounding body fluids. This is usually done via interactions with specific host cell receptors, therefore defining a \textit{receptor binding domain (RBD)} as the part of the surface protein that fits onto the receptor as a key fits into a lock.

At the same time, the human immune system will also get in contact with mostly surface proteins. The immune system will eventually develop antibodies that are targeted to bind to a specific region of virus surface proteins, thereby rendering them innocuous. The surface protein region tageted by the immune system is called \textit{epitope site} and may have a large overlap with the RBD.

These two mechanisms put an evolutionary pressure on the virus, especially on the surface proteins and its epitope sites. While having to maintain essentail funcionings as binding to receptors, the virus will draw large advatage from an epitope site modified to such an extent that the antibodies cannot bind to it any longer. This way of disguising itself\textemdash by animo acid mutations in crucial places\textemdash will allow the virus to reinfect previously immune individuals.

An accelerated evolution can be very beneficial to the virus, \textit{but}\textemdash since the capability to sequence and read out the viral genome\textemdash also provides a large record to trace a virus' history.

In this write-up, we will look at how the sequencing of the viral genome, along with phylogenetic tree inference, can provide useful insights on (i) the route that the virus takes to spread in the human population, (ii) the existence and prevalence of strains witin a virus (sub)type, and (iii) the projected evasion of the virus from immune system response and vaccines.



% 
\section{Influenza}

\subsection{Basics}

History diagram from Alberts, 2015

HA: epl

NA: expl

Phylo tree of A/H3N2 HA Volz+2013

\subsection{Hemagglutination Inhibition Assay}

Short description

Hirst 1943

mention cartography from Smith+2004 (?)

% \clearpage

% \section*{Appendix}





\end{document}
