\documentclass[12pt]{article}

\author{Paul Wiesemeyer}
\title{Phylogenetics for Predicting Virus Evolution}
\date{October 10, 2020}


\input{latex-includes/includes}
\usepackage[square]{natbib}

\usepackage{tikz}
\usetikzlibrary{trees}
\usetikzlibrary{shapes}
\usepackage{amsmath}
\usepackage{xspace}
\usepackage{forest}

\begin{document}

\maketitle


\section{Introduction}

  \subsection{Problem at Hand}

    Viruses and the diseases they provoke put a large burden on human life. These comparably simple structures are capable of modifying and possibly destroying vital functions in the human body, by infiltrating their genetic material into the reproductory apparatus of cells.

    All viruses that persist in the human population over longer times share some common features:
    \begin{itemize}
      \item They need some form of protection from their surroundings, a hull
      \item They need to find a way to get into human cells
      \item They aim at reproducing quickly
      \item If there is no constant infection source, they need to find a path to get from one individual to another
      \item They will have to deal with human immune system response
      \item They will underlie some kind of evolutionary pressure
    \end{itemize}

    (This rough characterisation is of course incomplete, and there may also be exceptions.)

    To give another rough image of what makes up a virus, here are the essential constituents:
    \begin{itemize}
      \item A piece of genetic code (RNA or DNA, positive or negative sense, usually some few kilobase(pairs) long)
      \item A hull: either only a capsid (of proteines) or an additional envelope (bilipid layer) with surface proteines
    \end{itemize}

    For enveloped viruses such as in the \textit{Orthomyxoviridae} family (including influenza) or \textit{Coronaviridae} (e.g. SARS-CoV-2), there is usually a variety of different surface proteins that populate the bilipid layer.

    These surface proteins take up functions such as binding to a cell to infiltrate the viruses genetic sequence into it, or releasing a freshly assembled virus from the host cell's surface into the surrounding body fluids. This is usually done via interactions with specific host cell receptors, therefore defining a \textit{receptor binding domain (RBD)} as the part of the surface protein that fits onto the receptor as a key fits into a lock.

    At the same time, the human immune system will also get in contact with mostly surface proteins. The immune system will eventually develop antibodies that are targeted to bind to a specific region of virus surface proteins, thereby rendering them innocuous. The surface protein region tageted by the immune system is called \textit{epitope site} and may have a large overlap with the RBD.

    These two mechanisms put an evolutionary pressure on the virus, especially on the surface proteins and its epitope sites. While having to maintain essentail funcionings as binding to receptors, the virus will draw large advatage from an epitope site modified to such an extent that the antibodies cannot bind to it any longer. This way of disguising itself\textemdash by animo acid mutations in crucial places\textemdash will allow the virus to reinfect previously immune individuals.

    An accelerated evolution can be very beneficial to the virus, \textit{but}\textemdash since the capability to sequence and read out the viral genome\textemdash also provides a large record to trace a virus' history.

    In this write-up, we will look at how the sequencing of the viral genome, along with phylogenetic tree inference, can provide useful insights on (i) the route that the virus takes to spread in the human population, (ii) the existence and prevalence of strains witin a virus (sub)type, and (iii) the projected evasion of the virus from immune system response and vaccines.

  \subsection{Influenza}

  Influenza is one of the most common viral diseases \citep{DudaMenna20} and puts a large burden on human health. The disease is estimated to cause symptomatic illness in about $3 - 11 \%$ of the human population (in the U.S.) \citep{tokarsOlsen+18} every season. Globally, up to $650,000$ individuals die from influenza illness every season \cite[see][]{iulianoRoguski+18}.

  Influenza refers to a disease caused by a virus of the influenza family \textit{Orthomyxoviridae}. All orthomyxo viruses have an envelope that carries surface proteines and their genome is a negative sense RNA.

  Most relevant for human infectious diseases are the genera \textit{Alpha-} and \textit{Betainfluenzavirus} that contain the species \textit{Influenza A} and \textit{Influenza B} respectively.\\[0mm]
% \forestset{
%   xlist/.style={
%     phantom,
%     for children={no edge,replace by={[,append,
%       delay={{asdf}}
%       ]}}
%   },
%   xlist/.default=0
%   }

\begin{tabular}{ c c c c c }

Taxonomy: & \hspace{18mm}Family\hspace{13mm} & Genus\hspace{3mm} & Species\hspace{10mm} & Serotype\hspace{1mm}\\
  
\end{tabular}\\[0mm]

\begin{forest}
  for tree={
    grow'=east,
    draw,
    calign=first,
    font=\sffamily,
    rounded corners,
    parent anchor=east,
    child anchor=west,
    edge path={%
      \noexpand\path [\forestoption{edge}] (!u.parent anchor) -- ++(0pt,0) |- (.child anchor)\forestoption{edge label};
    }
  },
  gray/.style={
    draw=none,
    color=gray
  }
  [{Viruses}
    [{$\cdot\cdot\cdot$}, gray
      [{\textit{Orthomyxoviridae}},for tree={l sep= 2mm}
        [{$\alpha$},for tree={l sep= 12mm}
          [{A},for tree={l sep= 18mm}
            [{H1N1}]
            [{H3N2}]
          ]
        ]
        [{$\beta$},for tree={l sep= 12mm}
          [{B},for tree={l sep= 18mm}
            [{Yamagata}]
            [{...}, gray]
          ]
        ]
        [{... (5 more)}, gray]
      ]
      [{\textit{... (149 more)}}, gray, for tree={l sep= 2mm}]
    ]
  ]
\end{forest}


  % \tikzstyle{every node}=[draw=black,thick,anchor=west]
  % \tikzstyle{selected}=[draw=red,fill=red!30]
  % \tikzstyle{optional}=[dashed]
  % \begin{tikzpicture}[%
  %   grow via three points={one child at (1.5,-0.4) and
  %   two children at (1.5,-0.4) and (1.5,-1.4)},
  %   edge from parent path={(\tikzparentnode.south) |- (\tikzchildnode.west)}]
  %   \node {Viruses}
  %     child { node [optional] {...}
  %       child { node {Orthomyxoviridae}
  %         child { node {$\alpha$}
  %           child { node {Influenza A}
  %             child { node {H1N1}}
  %             child { node {H3N2}}
  %           }
  %         }
  %         child { node {$\beta$}
  %                     child { node {Influenza B}}
  %         }
  %       }
  %     }
          
            
              
  %         %   }
  %         % }
          
  %         %   }
  %         % }
  %         child { node [optional] {...}};
  %       %   }

  %       % }
        
  %     % };
  % \end{tikzpicture}

% \begin{tikzpicture}[
%     grow=right,
%     level 1/.style={sibling distance=3.5cm,level distance=5.2cm},
%     level 2/.style={sibling distance=3.5cm, level distance=6.7cm},
%     edge from parent/.style={very thick,draw=blue!40!black!60,
%         shorten >=5pt, shorten <=5pt},
%     edge from parent path={(\tikzparentnode.east) -- (\tikzchildnode.west)},
%     kant/.style={text width=2cm, text centered, sloped},
%     every node/.style={text ragged, inner sep=2mm},
%     punkt/.style={rectangle, rounded corners, shade, top color=white,
%     bottom color=blue!50!black!20, draw=blue!40!black!60, very
%     thick }
%     ]

% \node[punkt, text width=5.5em] {Virus}
%     %Lower part lv1
%     child {
%         node[punkt] [rectangle split, rectangle split, rectangle split parts=3,
%          text ragged] {
%             \textbf{Scenario  1}
%                   \nodepart{second}
%             $\text{Country B}\colon    s\bar{Q}$
%                   \nodepart{third}
%             $\text{Country A}\colon\bar{Q}$
%         }
%         edge from parent
%             node[kant, below, pos=.6] {}
%     }
%     %Upper part, lv1
%     child {
%         node[punkt, text width=6em] {Country~A}
%         %child 1
%         child {
%             node [punkt,rectangle split, rectangle split,
%             rectangle split parts=3] {
%                 \textbf{Scenario  2}
%                 \nodepart{second}
%                 $\text{Country B}\colon s\bar{Q}+2\alpha\Delta E -sc$
%                 \nodepart{third}
%                 $\text{Country A}\colon\bar{Q}-\alpha\Delta E -
%                 \pa{1-s}c$
%             }
%             edge from parent
%                 node[below, kant,  pos=.6] {}
%         }
%         %child 2
%         child {
%             node [punkt, rectangle split, rectangle split parts=3]{
%                 \textbf{Scenario 3}
%                 \nodepart{second}
%                 $\text{Country B}\colon s\bar{Q}-2sc$
%                 \nodepart{third}
%                 $\text{Country A}\colon\bar{Q}-2\pa{1-s}c$
%             }
%             edge from parent
%                 node[kant, above] {}}
%             edge from parent{
%                 node[kant, above] {}}
%     };
% \end{tikzpicture}

    \subsubsection{Basics}

      History diagram from \citep{alberts15}

      HA: epl

      NA: expl

      Phylo tree of A/H3N2 HA Volz+2013

    \subsubsection{Hemagglutination Inhibition Assay}

      Short description

      Hirst 1943

      mention cartography from Smith+2004 (?)

\section{Phylodynamics}

  Essentially the Wikipedia Article + Volz 2013 (TODO read)


\section{Nextstrain}

  \subsection{Idea}

  \subsection{Short Usage Instructions}

  \subsection{Example}

\section{Mapping Influenza Evolution}

  \subsection{Mapping Titer to Tree}

    Minimizing a Cost function

    Tree model vs. substitution model

    Proving Treelikeness

  \subsection{Results}

    Figure 6 Interpretation

\section{Conclusion}

  Very interesting!


\bibliography{virophyle}
\bibliographystyle{apalike}

% 
\section{Introduction}

\subsection{Problem}

Viruses and the diseases they provoke put a large burden on human life. These comparably simple structures are capable of modifying and possibly destroying vital functions in the human body, by infiltrating their genetic material into the reproductory apparatus of cells.

All viruses that persist in the human population over longer times share some common features:
\begin{itemize}
  \item They need some form of protection from their surroundings, a hull
  \item They need to find a way to get into human cells
  \item They aim at reproducing quickly
  \item If there is no constant infection source, they need to find a path to get from one individual to another
  \item They will have to deal with human immune system response
  \item They will underlie some kind of evolutionary pressure
\end{itemize}

(This rough characterisation is of course incomplete, and there may also be exceptions.)

To give another rough image of what makes up a virus, here are the essential constituents:
\begin{itemize}
  \item A piece of genetic code (RNA or DNA, positive or negative sense, usually some few kilobase(pairs) long)
  \item A hull: either only a capsid (of proteines) or an additional envelope (bilipid layer) with surface proteines
\end{itemize}

For enveloped viruses such as in the \textit{Orthomyxoviridae} family (including influenza) or \textit{Coronaviridae} (e.g. SARS-CoV-2), there is usually a variety of different surface proteins that populate the bilipid layer.

These surface proteins take up functions such as binding to a cell to infiltrate the viruses genetic sequence into it, or releasing a freshly assembled virus from the host cell's surface into the surrounding body fluids. This is usually done via interactions with specific host cell receptors, therefore defining a \textit{receptor binding domain (RBD)} as the part of the surface protein that fits onto the receptor as a key fits into a lock.

At the same time, the human immune system will also get in contact with mostly surface proteins. The immune system will eventually develop antibodies that are targeted to bind to a specific region of virus surface proteins, thereby rendering them innocuous. The surface protein region tageted by the immune system is called \textit{epitope site} and may have a large overlap with the RBD.

These two mechanisms put an evolutionary pressure on the virus, especially on the surface proteins and its epitope sites. While having to maintain essentail funcionings as binding to receptors, the virus will draw large advatage from an epitope site modified to such an extent that the antibodies cannot bind to it any longer. This way of disguising itself\textemdash by animo acid mutations in crucial places\textemdash will allow the virus to reinfect previously immune individuals.

An accelerated evolution can be very beneficial to the virus, \textit{but}\textemdash since the capability to sequence and read out the viral genome\textemdash also provides a large record to trace a virus' history.

In this write-up, we will look at how the sequencing of the viral genome, along with phylogenetic tree inference, can provide useful insights on (i) the route that the virus takes to spread in the human population, (ii) the existence and prevalence of strains witin a virus (sub)type, and (iii) the projected evasion of the virus from immune system response and vaccines.



% 
\section{Influenza}

\subsection{Basics}

History diagram from Alberts, 2015

HA: epl

NA: expl

Phylo tree of A/H3N2 HA Volz+2013

\subsection{Hemagglutination Inhibition Assay}

Short description

Hirst 1943

mention cartography from Smith+2004 (?)

\end{document}
