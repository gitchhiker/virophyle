\newglossaryentry{hemagglutinationAssay}
{
    name=Hemagglutination Assay,
    description={A Lab experiment that classifies Influenza type A viruses according to their hemagglutinin surface protein}
}

\newglossaryentry{openScience}
{
    name=open science,
    description={The development to make scientific results, data and alike easily accessible to the public}
}

\newglossaryentry{phylogenetics}
{
    name=Phylogenetics,
    description={The science that extracts information from genetic sequences by classifying their inter-relatedness and inferring a \acrshort{ml} tree}
}

\newglossaryentry{phylodynamics}
{
    name=Phylodynamics,
    description={The branch of science that analyzes the various processes shaping phylogenies. These usually comprise evolutionary feedbacks as well as shorter timescale processes such as an extinction event}
}

\newglossaryentry{phylogeny}
{
    name=phylogeny,
    description={another word for phylogenetic tree}
}

\newglossaryentry{influenza}
{
    name=Influenza,
    description={The disease that is caused by one of the influenza viruses. Also called flu}
}

\newglossaryentry{evolution}
{
    name=Evolution,
    description={Systems that have a sense of succession and contain some form of \gls{heredity}, \gls{variation} and \gls{selection} show evolution. This can be as simple as ``survival of the fittest'' but also---depending on system complexity---a wide plethora of other dynamics}
}

\newglossaryentry{variation}
{
    name=variation,
    description={In \gls{evolution}, variation refers to the stochasticity of the properties that one generation inherits from its precursor}
}

\newglossaryentry{heredity}
{
    name=heredity,
    description={In \gls{evolution}, heredity means that one generation inherits traits from its preceding generation}
}

\newglossaryentry{selection}
{
    name=selection,
    description={In \gls{evolution}, selection is the feedback between fitness and probability to produce a next generation}
}

\newglossaryentry{strandedness}
{
    name=strandedness,
    description={In genetics, strandedness refers to whether a nucleic acid consists of one or two strands. The latter usually provides more stability, i.e. higher fidelity when copied}
}

\newglossaryentry{sense}
{
    name=sense,
    description={In genetics, the sense of a nucleic acid, indicates the roles of the strand (and its complement) in specifying a sequence of amino acids. In virology, positive-sense viral RNA means that the viral RNA sequence may be directly translated into viral proteins (5'-to-3')}
}

\newglossaryentry{capsid}
{
    name=capsid,
    description={The capsid of a virus is the protein shell that encloses the genetic material. It usually consists of many so called protomers that form an almost spherical structure (platonic body)}
}

\newglossaryentry{bilipidLayer}
{
    name=bilipid layer,
    description={The most ubiquitous membrane found in nature that consists of a double layer of lipid molecules}
}

\newglossaryentry{epitope}
{
    name=epitope,
    description={The part of an \gls{antigen}, such as a viral surface protein, that is recognized by the immune system and that antibodies bind to}
}

\newglossaryentry{antigen}
{
    name=antigen,
    description={Any kind of structure, that provokes a reaction by the immune system. See \gls{epitope}}
}

\newglossaryentry{immunology}
{
    name=immunology,
    description={Branch of science that deals with the immune system of organisms}
}

\newglossaryentry{epidemiology}
{
    name=epidemiology,
    description={Branch of science that studies the spreading and distribution of diseases and other health factors in populations. This can range from the abstract formulation of governing equations all the way to the exact description of a particular epidemic such as COVID-19}
}

\newglossaryentry{strain}
{
    name=strain,
    description={In biology, a strain refers to a subtype or a variant of some species.}
}

\newglossaryentry{species}
{
    name=species,
    description={In biology, ``species'' is the most widely used classification term with a range of definitions. Traditionally defined as ``largest group of organisms that are capable of producing fertile offspring'' it is used even beyond the realms of sexual reproduction with more sophisticated methods---e.g. based on genomic data---to distinguish between species}
}

\newglossaryentry{serotype}
{
    name=serotype,
    description={In biology, ``species'' is the most widely used classification term with a range of definitions. Traditionally defined as ``largest group of organisms that are capable of producing fertile offspring'' it is used even beyond the realms of sexual reproduction with more sophisticated methods---e.g. based on genomic data---to distinguish between species}
}

\newglossaryentry{titer}
{
    name=titer,
    description={SOMETHING SOMETHING}
}

\newglossaryentry{antigenicAdvancement}
{
    name=antigenic advancement,
    description={SOMETHING SOMETHING}
}

\newglossaryentry{antigenicDrift}
{
    name=antigenic drift,
    description={SOMETHING SOMETHING}
}

\newglossaryentry{antigenicShift}
{
    name=antigenic shift,
    description={SOMETHING SOMETHING}
}

\newglossaryentry{pointMutation}
{
    name=point mutation,
    description={SOMETHING SOMETHING}
}

\newglossaryentry{indel}
{
    name=indel,
    description={SOMETHING SOMETHING}
}

\newglossaryentry{allele}
{
    name=allele,
    description={SOMETHING SOMETHING}
}

\newacronym{rbd}{RBD}{Receptor Binding Domain}

\newacronym{hi}{HI}{Hemagglutination Inhibition}

\newacronym{ha}{HA}{Hemagglutinin}

\newacronym{na}{NA}{Neuraminidase}

\newacronym{rna}{RNA}{Ribonucleic Acid}

\newacronym{dna}{DNA}{Deoxyribonucleic Acid}

\newacronym{ml}{ML}{Maximum Likelihood}



